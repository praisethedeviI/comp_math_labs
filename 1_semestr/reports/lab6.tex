\documentclass[14pt, a4paper, fleqn]{extarticle}
\usepackage[russian]{babel}
\usepackage[utf8]{inputenc}
\usepackage{amsmath, mathtools}
\usepackage{multicol}
\usepackage{tabularx}

\usepackage{graphicx}

\usepackage{style}

\begin{document}
	\createtitle{6}
	\tableofcontents
	\pagebreak
	\makesection{Метод монотонной прогонки}
	\makesubsection{Постановка задачи}
	Требуется найти решение системы линейных алгебраических уравнений (СЛАУ). При этом СЛАУ имеет специальный вид:
	\[
	A_kx_{k-1} + B_kx_k + C_kx_{k+1} = F_k, \qquad k = 1,...,N  \qquad A_1 = C_N = 0
	\]
	\makesubsection{Решение}
	Будем решать данную СЛАУ:
	\[
	A = \begin{pmatrix}
		2 & 2 & 0  \\
		1 & 2 & 1  \\
		1 & 2 & 1  \\
		0 & 2 & 1  \\
	\end{pmatrix} \qquad
	\vec{b} = \begin{pmatrix}
		2 \\
		4 \\
		4 \\
		3 \\
	\end{pmatrix}
	\]
	Для начала найдем все $\alpha$ и $\beta$, исходя из формул:
	\begin{multline*}
		\alpha_2 = -\dfrac{C_1}{B_1} \qquad \beta = \dfrac{F_1}{B_1} \\
		\alpha_k = - \dfrac{C_k}{B_k + A_k\alpha_k} \qquad \beta = \dfrac{F_k - A_k\beta_k}{B_k + A_k\alpha_k}\\
	\end{multline*}
	Предположим, что:
	\[
	x_{k-1} = \alpha_kx_k + \beta_k
	\]
	Тогда решения будем искать по следующим формулам:
	\begin{multline*}
		x_k = \alpha_kx_{k-1} + \beta_k \\
	\end{multline*}
	Чтобы найти переменную в $N$ позиции будем использовать данную формулу:
	\[
	x_N = \dfrac{F_N - A_N\beta_N}{A_N\alpha_N + B_N}
	\]
	Следуя из всех формул выше, мы можем найти ответ СЛАУ:
	\[
	\vec{x} = 
	\begin{pmatrix}
		2\\
		-1\\
		4\\
		-3\\
	\end{pmatrix}
	\]
	\makesection{Приложения}
	\begin{lstlisting}[language=Python, caption={Реализация метода прогонки}]
		low = [0, 1, 1, 2]
		main = [2, 2, 2, 2]
		high = [1, 1, 1, 0]
		result = [3, 4, 4, 2]
		
		alpha = [0]
		betta = [0]
		alpha.append(-high[0] / main[0])
		betta.append(result[0] / main[0])
		
		n = 4
		i = 1
		
		while i < n:
			alpha.append(-high[i] / (main[i] + low[i] * alpha[i]))
			betta.append((-low[i] * betta[i] + result[i]) / (main[i] + low[i] * alpha[i]))
			i += 1
		
		print(alpha)
		print(betta)
		
		x = [(-low[n - 1] * betta[n - 1] + result[n - 1]) / (main[n - 1] + low[n - 1] * alpha[n - 1])]
		i = n - 1
		x_count = 0
		
		while i != 0:
			x.append(alpha[i] * x[x_count] + betta[i])
			i -= 1
			x_count += 1
		
		x.reverse()
		print(x)
	\end{lstlisting}
	\makesection{Вывод}
	В данной лабораторной работе было произведено вычисление решения СЛАУ с помощью метода прогонки.
			
\end{document}
