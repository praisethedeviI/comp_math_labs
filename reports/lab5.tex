\documentclass[14pt, a4paper, fleqn]{extarticle}
\usepackage[russian]{babel}
\usepackage[utf8]{inputenc}
\usepackage{amsmath, mathtools}
\usepackage{multicol}
\usepackage{tabularx}

\usepackage{graphicx}

\usepackage{style}

\begin{document}
	\createtitle{5}
	\tableofcontents
	\pagebreak
	\makesection{Введение}	
	Интегрирование функции с помощью формулы Гаусса. Вариант 16
	\makesection{Численное интегрирование}
	\makesubsection{Постановка задачи}
	Провести численную интегрирование с помощью формулы Гаусса:
	\[
	I_n = \dfrac{b - a}{2} \sum_{k=1}^{n}c_kf(x_k) \qquad x_k = \dfrac{b+a}{2} + \dfrac{b - a}{2} \cdot t_k
	\]
	\makesubsection{Решение}
	В случае варианта 16 имеем следующие условия:
	\begin{multline*}
		n = 4, \qquad t_{1, 4} = \pm 0.861136 \qquad t_{2, 3} = \pm 0.339981 \\
		c_{1, 4} = 0.347855 \qquad c_{2, 3} = 0.652145\\
	\end{multline*}
	Разделим отрезок $[a, b]$ сначала на 2 части, потом 4 части и сравним их:
	\[
	I_{2n}(a, b) = I_n(a, m) + I_n(m, b) \qquad m = \dfrac{a + b}{2}
	\]
	\begin{multline}
		I_{4n} = I_{2n}(a, m) + I_{2n}(m, b) = I_n(a, d) + I_n(d, m) + I_n(m, c) + I_n(c, b)\\
		d = \dfrac{a + m}{2} \quad c = \dfrac{m + b}{2}\\
	\end{multline}
	Получаем следующие значения: $I_{4n} = 1.1661089745138693$ и $I_{2n} = 1.1661089738845032$. Сравнивая разницу между ними получаем: $\Delta I = 6.293661147793728e-10$	

	\makesection{Приложения}
	\begin{lstlisting}[language=Python, caption={Реализация численного интегрирования}]
		import math
		
		n = 4
		t = [
			0.861136,
			0.339981,
			-0.339981,
			-0.861136
		]
		c = [
			0.347855,
			0.652145,
			0.652145,
			0.347855
		]
		a = 0.5
		b = 1.0
		
		
		def func(x):
			return x ** 2 + math.log(x + 5)
		
		
		def find_x(a, b, t):
			return (b + a) / 2 + (b - a) * t / 2
		
		
		def find_I(a, b):
			return ((b - a) / 2) * (c[0] * func(find_x(a, b, t[0])) +
									c[1] * func(find_x(a, b, t[1])) +
									c[2] * func(find_x(a, b, t[2])) +
									c[3] * func(find_x(a, b, t[3])))
		
		
		def main():
			I_n = find_I(a, b)
			m = (a + b) / 2
			I_2n = find_I(a, m) + find_I(m, b)
			eps = 0.000001
			I_4n = find_I(a, (a + m) / 2) + find_I((a + m) / 2, m) + find_I(m, (m + b) / 2) + find_I((m + b) / 2, b)
			print('I_n =',I_n)
			print('I_2n =',I_2n)
			print('I_4n =',I_4n)
			print('I_4n - I_2n =',I_4n - I_2n)
		
		
		if __name__ == '__main__':
			main()
	\end{lstlisting}
	\makesection{Вывод}
	В данной лабораторной работе было произведено вычисление интеграла с помощью формулы Гаусса.
		
\end{document}
