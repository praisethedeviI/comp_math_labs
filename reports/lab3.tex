\documentclass[14pt, a4paper, fleqn]{extarticle}
\usepackage[russian]{babel}
\usepackage[utf8]{inputenc}
\usepackage{amsmath, mathtools}
\usepackage{multicol}
\usepackage{tabularx}

\usepackage{graphicx}

\usepackage{style}

\begin{document}
	\createtitle{3}
	\tableofcontents
	\pagebreak
	\makesection{Введение}
	Интерполирование функции с помощью интерполяционных формул c конечными
	разностями. Вариант №16
	\makesection{Интерполирование функции}
	\makesubsection{Постановка задачи}
	При заданных условиях построить таблицу конечных разностей, использовать формулы Ньютона и Гаусса, найти остаточный член и проверить неравенства.
	
	Нам заданы следующие условия:
	\[
	y = x^2 + \ln(x + 5) \quad [a, b] = [0.5, 1.0]
	\]
	\[
	x^{**} = 0.52 \quad x^{***} = 0.97 \quad x^{****} = 0.73
	\]
	
	\makesubsection{Использование интерполяционных формул}
	Построим таблицу конечных разностей и найдем значение функций по интерполяционным формулам.
	
	Так как $x^{**} = 0.52$ находится в начале таблицы, то мы будем использовать первую формулу Ньютона:
	\[
	N_1 = f_0 + tf^1_{\frac{1}{2}} + \dfrac{t(t-1)}{2}f^2_1 + ... + \dfrac{t(t-1)...(t-n+1))}{n!}f^n_{\frac{n}{2}}
	\]
	
	$x^{***} = 0.97$ находится в конце таблицы, используем вторую формулу Ньютона:
	\[
	N_2 = f_0 + tf^1_{-\frac{1}{2}} + \dfrac{t(t+1)}{2}f^2_{-1} + ... + \dfrac{t(t+1)...(t+n-1))}{n!}f^n_{-\frac{n}{2}}
	\]
	
	$x^{****} = 0.73$ находится в середине таблицы и ближе к правому значению, используем вторую формулу Гаусса для интерполирования назад:
	\[
	G_2 = f_0 + tf^1_{-\frac{1}{2}} + \dfrac{t(t+1)}{2}f^2_{0} + ... + \dfrac{t\big(t^2-1\big)...\big(t^2-(\frac{n}{2}-2)^2\big)\big(t + (\frac{n}{2}-1)\big)}{n!}f^n_{-\frac{n}{2}}
	\]
	
	При этом $t$ будем искать, исходя из условия: $x = x_0 + th$. Получаем значения:
	\begin{multline*}
	N_1 \approx 1.9787778602890034 \\
	N_2 \approx 2.7276469274045083 \\
	G_2 \approx 2.279600354953445 \\ 
	\end{multline*}
	\makesubsection{Нахождение отсточного члена}
	
	Необходимо оценить минимальное и максимальное значения $f^{(n+1)}(x)$, а также минимальное и максимальное значения остаточного члена $R_n(x)$.
	
	Остаточный член будем находить по формуле:
	\begin{multline*}
	R_n(x) = \dfrac{f^{(n+1)}(\xi)}{(n+1)!}\omega_{n+1}(x), \quad \omega = (x - x_0)...(x-x_n), \quad \xi \in [a, b] \\
	\end{multline*}
	
	Одиннадцатые производные от $x$:
	\begin{multline*}
	\dfrac{d^{11}f}{dx^{11}} = \dfrac{10!}{(x+5)^{11}} \\
	\dfrac{d^{11}f}{dx^{11}} \approx 0.02604794393473182, \quad x = 0.5\\
	\dfrac{d^{11}f}{dx^{11}} \approx 0.010002286236854138, \quad x = 1.0\\
	\end{multline*}

	Найдем остаточные члены для $x^{**}$:
	\begin{multline*}
	R_{max}(x) \approx 1.2217094185359548	e-18 \\
	R_{min}(x) \approx 4.691305898491087e-19 \\
	R(x^{**}) \approx 1.1738906714157153e-18 \\
	\end{multline*}

	Найдем остаточные члены для $x^{***}$:
	\begin{multline*}
		R_{max}(x) \approx 7.762158094748162e-19 \\
		R_{min}(x) \approx 2.9806316872427965e-19 \\
		R(x^{***}) \approx 3.1495929230145176e-19 \\
	\end{multline*}

	Найдем остаточные члены для $x^{****}$:
	\begin{multline*}
		R_{max}(x) \approx 1.429938349418616e-20 \\
		R_{min}(x) \approx 5.4908950617284105e-21 \\
		R(x^{****}) \approx 9.111839398100736e-21 \\
	\end{multline*}
	
	\makesection{Приложения}
	Код программы, вычисляющей значения формул, приложен отдельным файлом.
	
	\makesection{Вывод}
	В данной лабораторной работе были проделаны вычисления, произведенные по интерполяционным формулам с конечными разностями, для нахождения приблизительного значения функции; а также вычислены остаточные члены, чтобы узнать погрешность данных формул.
	
	
\end{document}