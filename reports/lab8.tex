\documentclass[14pt, a4paper, fleqn]{extarticle}
\usepackage[russian]{babel}
\usepackage[utf8]{inputenc}
\usepackage{amsmath, mathtools}
\usepackage{multicol}
\usepackage{tabularx}

\usepackage{graphicx}

\usepackage{style}

\begin{document}
	\createtitle{7}
	\tableofcontents
	\pagebreak
	\makesection{Метод хорд и касательных}
	\makesubsection{Постановка задачи}
	Вариант 16. Необходимо найти корни данного уравнения численным методом хорд:
	\[
	0 = 0.5x^2 - \cos(2x)
	\]
	\makesubsection{Решение}
	Воспользуемся расчетной формулой метода хорд, которая имеет вид:
	\[
	x_n = x_{n-1} - \dfrac{f(x_{n-1})}{f'(x_{n-1})}
	\]
	Производная от данной нам функции $y = 0.5x^2 - \cos(2x)$ имеет вид:
	\[
	y' = x + 2\sin(2x)
	\]
	Найдем корни уравнения с помощью средств компьютерной математики:
	\[
	x_{1, 2} \approx \pm 0.6716432093903656
	\]
	
	\makesection{Приложения}
	\begin{lstlisting}[language=Python, caption={Реализация метода хорд}]
		import math
		
		
		def func(x):
			return 0.5 * x ** 2 - math.cos(2 * x)
		
		
		def dif(x):
			return x + 2 * math.sin(2*x)
		
		
		def main():
			x = [5]
			for i in range(100):
				x.append(x[-1] - func(x[-1])/ dif(x[-1]))
			print(x[-1])
		
			x = [-5]
			for i in range(100):
				x.append(x[-1] - func(x[-1]) / dif(x[-1]))
			print(x[-1])
		
		
		if __name__ == '__main__':
			main()
	\end{lstlisting}
	\makesection{Вывод}		
	В данной лабораторной работе с помощью средств компьютерной математики нашли численным методом хорд корни уравнения.
	
\end{document}