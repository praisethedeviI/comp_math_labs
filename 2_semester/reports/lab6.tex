\documentclass[14pt, a4paper, fleqn]{extarticle}
\usepackage[russian]{babel}
\usepackage[utf8]{inputenc}
\usepackage{amsmath, mathtools}
\usepackage{multicol}
\usepackage{tabularx}

\usepackage{graphicx}

\usepackage{style}

\begin{document}
	\createtitle{6}
	\tableofcontents
	\pagebreak
	\makesection{Введение}	
	В данной лабораторной работе необходимо найти обратную матрицу с помощью метода окаймления.
	\makesection{Постановка задачи}
	Дана матрица $A$. Необходимо найти матрицу $A^{-1}$ методом окаймления.
	\makesection{Алгоритм работы}	
	\begin{enumerate}
		\item Представим матрицу $A = A_n$. Данную матрицу мы представляем в следующем виде:
		$$A_n = \begin{pmatrix}
		A_{n-1} & u_n \\
		\upsilon_n  & a_{nn}	
		\end{pmatrix}$$
		$\upsilon_n = (a_{n1}, a_{n2},  ..., a_{nn-1})$, $u_n = (a_{1n}, a_{2n},  ..., a_{n-1n})$, $A_{n-1}$ -- матрица $n-1$-го порядка.
		\item Матрицу $A^{-1}$ также будем представлять окаймленной:
		$$A_n^{n-1} = \begin{pmatrix}
			P_{n-1} & r_n \\
			q_n  & \dfrac{1}{\alpha_n}	
		\end{pmatrix}$$.
		\item Коэффициенты окаймленной обратной матрицы будет искать по следующим формулам:
		$$r_n = -\dfrac{A^{-1}_{n-1}u_n}{\alpha_n}, ~~~ \alpha_n = a_{nn} - \upsilon_nA^{-1}_{n-1}u_n, $$
		$$P_{n-1} = A^{-1}_{n-1} - A^{-1}_{n-1}u_nq_n, ~~~ q_n = -\dfrac{\upsilon_nA^{-1}_{n-1}}{\alpha_n} $$.
		
		\item Таким образом, получаем такую матрицу:
		$$
		A^{-1} = \begin{pmatrix}
			A^{-1}_{n-1} - \dfrac{A^{-1}_{n-1}u_n\upsilon_nA^{-1}_{n-1}}{a_{nn} - \upsilon_nA^{-1}_{n-1}u_n} & -\dfrac{A^{-1}_{n-1}u_n}{a_{nn} - \upsilon_nA^{-1}_{n-1}u_n} \\
			-\dfrac{\upsilon_nA^{-1}_{n-1}}{a_{nn} - \upsilon_nA^{-1}_{n-1}u_n}  & \dfrac{1}{a_{nn} - \upsilon_nA^{-1}_{n-1}u_n}	
		\end{pmatrix}
		$$
	\end{enumerate}
	
	
	\makesection{Приложения}
	
	\begin{lstlisting}[language=Python, caption={Компьютерная реализация алгоритма}]
	import numpy as np
	
	
	def get_inv(A, depth=0):
		n = len(A)
		k = n - 1
	
		if n == 1:
			return np.matrix([[1 / A[0, 0]]])
	
		Ap = A[:k, :k]
		V, U = A[k, :k], A[:k, k]
	
		Ap_inv = get_inv(Ap, depth + 1)
	
		alpha = 1 / (A[k, k] - V * Ap_inv * U).item()
		Q = -V * Ap_inv * alpha
		P = Ap_inv - Ap_inv * U * Q
		R = - Ap_inv * U * alpha
	
		A_inv = np.matrix([[0.0] * n for _ in range(n)])
		A_inv[:k, :k] = P
		A_inv[k, :k] = Q[0]
	
		A_inv[:k, k] = R[:, 0]
		A_inv[k, k] = alpha
	
		return A_inv
	
	
		def main():
		mat = np.matrix([
			[1, 2, 4],
			[3, 3, 2],
			[4, 1, 3]
		])
	
		print(get_inv(mat))
	
	
	if __name__ == '__main__':
		main()
	\end{lstlisting}
	\makesection{Вывод}
	В данной лабораторной работе была произведена реализация вычисления обратной матрицы методом окаймления.
\end{document}
