\documentclass[14pt, a4paper, fleqn]{extarticle}
\usepackage[russian]{babel}
\usepackage[utf8]{inputenc}
\usepackage{amsmath, mathtools}
\usepackage{multicol}
\usepackage{tabularx}

\usepackage{graphicx}

\usepackage{style}

\begin{document}
	\createtitle{5}
	\tableofcontents
	\pagebreak
	\makesection{Введение}		
	В данной лабораторной работе необходимо найти решение уравнения с помощью QR-разложения матрицы
	
	\makesection{Постановка задачи}
	Дана матрица $A$, необходимо ее разложить на $A = QR$ и решить СЛАУ $Ax = B$.
	
	\makesection{Алгоритм работы}
	\begin{enumerate}
		\item С помощью QR-разложения мы можем представить матрицу $A$ в виде $Q \cdot R$, где $Q$ -- ортогональная матрица ($QQ^T = I, Q^T = Q^{-1}$), а $R$ -- верхнеугольная матрица.
		\item Чтобы найти матрицу $Q$, найдем матрицу $P = I - 2\dfrac{p, x}{p, p}p$, где $p$ является вектором нормали.
		\item Матрицу $R$ несложно вычислить, зная $A = QR$, следовательно матрица вычислятся по формуле $R = Q^TA$.
		\item Решение СЛАУ будет следующим: $x = R^{-1}Q^TB$.	 	
	\end{enumerate}
	\makesection{Приложения}
	\begin{lstlisting}[language=Python, caption={Компьютерная реализация алгоритма}]
		import numpy as np
		from lab2 import SLE
		
		
		def QR(A):
			def step(a):
				v = a / (a[0] + np.copysign(np.linalg.norm(a), a[0]))
				v[0] = 1
				P = np.eye(a.shape[0])
				P -= (2 / np.dot(v, v)) * np.dot(v[:, None], v[None, :])
				return P
		
			m, n = A.shape
			Q = np.eye(m)
			for i in range(n):
				P = np.eye(m)
				P[i:, i:] = step(A[i:, i])
				Q = np.dot(Q, P)
				A = np.dot(P, A)
			return np.matrix(Q), np.matrix(A)
		
		
		def solve(A, b):
			Q, R = QR(A)
			Q = Q.tolist()
			sle = SLE(Q, b.tolist())
			sle.solve()
			y = sle.get_vec()
			sle = SLE(R.tolist(), y)
			sle.reverse()
			x = sle.get_vec()
			return x
		
		
		def main():
			mat = np.array((
				[1, 2, 4],
				[3, 3, 2],
				[4, 1, 3]
			), dtype=float)
		
			vec = np.array([4, 2, 3])
		
			print(solve(mat, vec.copy()))
		
			sle = SLE(mat.tolist(), vec.tolist())
			print(sle.solve()[1])
		
		
		if __name__ == '__main__':
			main()
		
	\end{lstlisting}
	\makesection{Вывод}
	В данной лабораторной работе было произведено вычисление решения СЛАУ с помощью QR-разложения.
\end{document}
