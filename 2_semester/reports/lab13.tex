\documentclass[14pt, a4paper, fleqn]{extarticle}
\usepackage[russian]{babel}
\usepackage[utf8]{inputenc}
\usepackage{amsmath, mathtools}
\usepackage{multicol}
\usepackage{tabularx}

\usepackage{graphicx}

\usepackage{style}

\begin{document}
	\createtitle{13}
	\tableofcontents
	\pagebreak
	\makesection{Введение}	
	В данной лабораторной работе необходимо найти собственные значения матрицы с помощью метода вращений.
	\makesection{Постановка задачи}
	Дана матрица $A$. Необходимо найти все собственные значения $\lambda$ данной матрицы методом вращений.
	\makesection{Алгоритм работы}	
		\begin{enumerate}
			\item Найдем коэффициенты максимального абсолютно	 элемента матрицы, учитывая $i < j$. Коэффициенты обозначим за $l$ и $m$.
			\item Построим единичную матрицу такого же размера, как и матрица $A$, но при этом определим на $l$ и $m$ элементы. Матрица примет такой вид:
			$$ T = 
			\begin{pmatrix}
				1 & ... & ... & ... & ... & ... & 0 \\
				... & ... & ... & ... & ... & ... & ... \\
				... & ... & \cos \varphi & ... & -\sin \varphi & ... & ... \\
				... & ... & ... & ... & ... & ... & ... \\
				... & ... & \sin \varphi & ... & \cos \varphi & ... & ... \\
				... & ... & ... & ... & ... & ... & ... \\
				0 & ... & ... & ... & ... & ... & 1 \\
			\end{pmatrix}
			$$
			\item Найдем угол $\varphi = \dfrac{\arctan\left( \frac{2a_{lm}}{a_{ll} - a{mm}} \right) }{2} $.
			\item Вычислим $A^{(k+1)} = T^TA^{(k)}T $.
			\item По итогу, через некоторое количество итераций  $k = 1, 2, 3....$(В нашем случае $k  = 15$) на главной диагонали матрицы $A^{(k)}$ будут располагаться приближенные к точному собственные значения матрицы $A$.
		\end{enumerate}
	\makesection{Приложения}
	
	\begin{lstlisting}[language=Python, caption={Компьютерная реализация алгоритма}]
		import numpy as np
		
		
		def get_max(A):
			index = (1, 0)
			for i in range(A.shape[0]):
				for j in range(i):
					if abs(A[index]) < abs(A[i, j]):
						index = (i, j)
			return index
		
		
		def rotation_method(A):
		
			for k in range(15):
				l, m = get_max(A)
				phi = np.arctan(2 * A[l, m] / (A[l, l] - A[m, m])) / 2
				T = np.eye(A.shape[0])
				T[l, l] = np.cos(phi)
				T[l, m] = -np.sin(phi)
				T[m, l] = np.sin(phi)
				T[m, m] = np.cos(phi)
		
				A = T.T * A * T
		
			return A
		
		
		def main():
			vec = np.matrix([
				[-0.168700, 0.353699, 0.008540, 0.733624],
				[0.353699, 0.056519, 0.723182, -0.076440],
				[0.008540, 0.723182, 0.015938, 0.342333],
				[0.733624, -0.076440, 0.342333, -0.045744]
			])
			np.set_printoptions(precision=4, suppress=True)
		
			print(rotation_method(vec).diagonal())
		
		
		if __name__ == '__main__':
			main()
	\end{lstlisting}
	\makesection{Вывод}
	В данной лабораторной работе был реализован поиск собственных значений с помощью метода вращений.
\end{document}
