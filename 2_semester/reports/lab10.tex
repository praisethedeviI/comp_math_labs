\documentclass[14pt, a4paper, fleqn]{extarticle}
\usepackage[russian]{babel}
\usepackage[utf8]{inputenc}
\usepackage{amsmath, mathtools}
\usepackage{multicol}
\usepackage{tabularx}

\usepackage{graphicx}

\usepackage{style}

\begin{document}
	\createtitle{10}
	\tableofcontents
	\pagebreak
	\makesection{Введение}	
	В данной лабораторной работе необходимо найти решение СЛАУ с помощью метода Ричардсона.
	\makesection{Постановка задачи}
	Дана матрица $A$ и вектор $B$. Необходимо решить систему $Ax = B$ с помощью итерационного метода Ричардсона.
	\makesection{Алгоритм работы}	
	\begin{enumerate}
		\item Метод Ричардсона является ускоренным методом простых итераций за счет параметра $t$, вычисляемый по формуле: $$t = \dfrac{2}{(m + M) + (M + m)\cos\left( \frac{(2i-1)\pi}{2k}\right) }$$.
		\item В вычислении параметра $t$ присутствуют значения $m$ и $M$. Данные значения являются собственными у матрицы $A$ минимальным и максимальным соответственно.	
		\item Формула для вычисления решения имеет следующий вид:
		$$
		x^{(k+1)} = t(B - Ax^{(k)})
		$$.
		\item Используя данный метод, установим максимальное количество итераций. В нашем случае $k = 16$.
	\end{enumerate}
	\pagebreak
	\makesection{Приложения}
	
	\begin{lstlisting}[language=Python, caption={Компьютерная реализация алгоритма}]
		from numpy import linalg
		import math
		
		
		def richardson(mat, vec, k):
			a, b = __find_min_max_eig(mat)
			dim = len(vec)
			x = [[0 for _ in range(dim)]]
		
			it = 0
			while it < k:
				it += 1
				x.append([])
				t = 2 / ((a + b) + (b - a) * math.cos((2 * it - 1) * math.pi / (2 * k)))
				for i in range(dim):
					x[it].append(x[it - 1][i] -
									__find_subtraction(x[it - 1], mat, vec, i) * t)
		
			return x[-1], it
		
		
		def __find_min_max_eig(mat):
			eig = linalg.eig(mat)[0]
			return min(eig), max(eig)
		
		
		def __find_subtraction(x, mat, vec, pos):
			sub = -vec[pos]
			for j in range(len(vec)):
				sub += mat[pos][j] * x[j]
			return sub
		
		
		def main():
			mat = [
				[10, 3, 3, 1, 2, 1],
				[1, 14, 1, 3, 5, 1],
				[2, 1, 7, 1, 3, 2],
				[1, 2, 2, 12, 3, 1],
				[1, 4, 2, 2, 9, 1],
				[2, 1, 2, 5, 4, 8],
			]
			vec = [4, 1, 4, 6, 3, 6]
		
			print(richardson(mat, vec, 16)[0])
			print(linalg.solve(mat, vec))
		
		
		if __name__ == '__main__':
			main()
	\end{lstlisting}
	\makesection{Вывод}
	В данной лабораторной работе был реализован итерационный метод Ричардсона.
\end{document}
	